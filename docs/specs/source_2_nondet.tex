\input source_header.tex

\begin{document}
	%%%%%%%%%%%%%%%%%%%%%%%%%%%%%%%%%%%%%%%%%%%%%%%
	\docheader{2021}{Source}{\S 2 Non-Det}{Martin Henz, Zhao Mingda, Cai Xiaolin}
	%%%%%%%%%%%%%%%%%%%%%%%%%%%%%%%%%%%%%%%%%%%%%%%

\input source_intro.tex

Source \S 2 Non-Det is a nondeterministic variant of Source \S 2.

\section*{Changes}

% some of the description is taken from the textbook

Source \S 2 Non-Det supports a programming paradigm called nondeterministic computing by building into the evaluator a facility to support automatic search. \newline

Nondeterministic program evaluator will work by automatically choosing a possible value and keeping track of the choice. If a subsequent requirement is not met, the evaluator will try a different choice, and it will keep trying new choices until the evaluation succeeds, or until we run out of choices.\newline

In order to support nondeterminism, we implement a new expression \textbf{\texttt{amb}}.
This expression returns the values in parameter to the caller one by one as alternatives,
and returns \textbf{\texttt{undefined}} when running out of choices.\newline

A special command \textbf{\texttt{try\_again();}} can be executed in REPL to force trying the next alternative.\newline


\input source_bnf.tex

\newpage

\input source_2_bnf.tex

\newpage

\input source_boolean_operators

\input source_return

\input source_names

\input source_nondet_2

\input source_lists

\input source_numbers

\input source_strings

\input source_typing

\input source_comments

\input source_js_differences

\newpage

\input source_list_library

    \end{document}
